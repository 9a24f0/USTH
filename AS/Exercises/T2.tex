\documentclass{article}

\usepackage[utf8]{inputenc}
\usepackage{mathtools}
\usepackage{amssymb}
\usepackage{graphicx}
\usepackage{ragged2e}
\usepackage{listings}
\usepackage[a4paper, total={6in, 8in}]{geometry}

\newcommand{\Z}{\mathbb{Z}}
\newcommand{\N}{\mathbb{N}}
\newcommand{\Q}{\mathbb{Q}}
\newcommand{\R}{\mathbb{R}}

\begin{document}
    \centering
	\LARGE\textbf{Tutorial \#2: Mapping}
	
	\justifying
	\Large
    \section{Exercise 1:}
        \begin{equation}
            f(p/q) = \frac{p+1}{p-2}
        \end{equation}
        When $p = 2, \forall q \in \Q^*$, we have: $f(2/q) =  \frac{3}{0}$.\\
        There exist $(p/q)$ in domain such that $f(p/q)$ in codomain doesn't exist. Thus, (1) is not mapping. 
        \begin{equation}
            f(p/q) = \frac{p}{p+q}
        \end{equation}
        When $p = -q$, we have: $f(-1) =  \frac{p}{0}$.\\
        There exist $(p/q)$ in domain such that $f(p/q)$ in codomain doesn't exist. Thus, (2) is not mapping.
	\section{Exercise 2:}
		\begin{enumerate}
			\item[a)]
				\begin{equation*}
					f: \R \rightarrow \R \mbox{ defined by } f(x) = e^x
				\end{equation*}
				As $f'(x) = e^x > 0 \; \forall x \in \R$, the function $f$ only increases in $\R$. Thus, the function is injective.
				However, for negative elements in codomain $(f(x) < 0)$, there exists no $x \in \R$ maps to $f(x)$. Hence, the function injective but not subjective.
			\item[b)]
				\begin{equation*}
					f: \Z \rightarrow \Z \mbox{ defined by } f(n) = n^2 + 5
				\end{equation*}
				As $f(-n) = f(n) \; \forall n \in \Z$, the function is not injective. In addition, when $f(n) = 3$, there no such $x \in \Z$ maps to $f(x)$. Thus, the function is neither subjective nor injective.
			\item[c)]
				\begin{equation*}
					f: \R \rightarrow \R \mbox{ defined by } f(x) = sin(x)
				\end{equation*}
				As $f(2\pi) = f(0) = 1$, the function is not injective. In addition, when $f(n) = 2$, there no such $x \in \Z$ maps to $f(x)$. Thus, the function is neither subjective nor injective.
		\end{enumerate}
		
	\section{Exercise 3:}
		\begin{enumerate}
			\item [a)]
				\begin{equation*}
					f(x) = \frac{1}{2}x + 7
				\end{equation*}
				Let $y = f(x)$, we re-write the equation as
				\begin{equation*}
					y = \frac{1}{2}x + 7 \iff x = 2y - 14
				\end{equation*}
				Thus, the inverse function is $f^{-1}(x) = 2x -14$
			\item[b)]
				\begin{equation*}
					f(x) = (x-2)^3 + 1
				\end{equation*}
				let $y = f(x)-1$, we re-write the equation as
				\begin{equation*}
					y = (x-2)^3 \iff x = \sqrt[3]{y} +2
				\end{equation*}
				Thus, the inverse function is $f^{-1}(x) = \sqrt[3]{x-1} +2$
			\item[c)]
				\begin{equation*}
					f(x) = \frac{1+2x}{7+x}
				\end{equation*}
				Let $y = f(x)$, we re-write the equation as
				\begin{equation*}
					y = \frac{1+2x}{7+x} \iff x = \frac{1-7y}{y-2}
				\end{equation*}
				Thus, the inverse function is $f^{-1}(x) = \frac{1-7x}{x-2}$
		\end{enumerate}
	
	\section{Exercise 4:}
		The plot leaves for reader ;)\\
		The domain of $f$: $ x \in \R \setminus \{0\}.$ The range of $f$: $f \in \R \setminus \{0\}$\\
		Let $y = f(x)$, we re-write the equation as
		\begin{equation*}
			y = \frac{x+1}{x-1} \iff x = \frac{y+1}{y-1}
		\end{equation*}
		Thus, the inverse function of $f$ is $f^{-1} = \frac{x+1}{x-1}$.\\
		\textbf{Compare:}
		\begin{equation*}
			f \circ f^{-1} =f(f^{-1}(x)) = \frac{\frac{x+1}{x-1} + 1}{\frac{x+1}{x-1}-1} = x
		\end{equation*}
		\begin{equation*}
			f^{-1} \circ f =f^{-1}(f(x)) = \frac{\frac{x+1}{x-1} + 1}{\frac{x+1}{x-1}-1} = x
		\end{equation*}
		\textbf{Conclusion:} $f \circ f^{-1} = f^{-1} \circ f$ 
		
\end{document}