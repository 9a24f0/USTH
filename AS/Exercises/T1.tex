\documentclass{article}

\usepackage[utf8]{inputenc}
\usepackage{mathtools}
\usepackage{amssymb}
\usepackage{graphicx}
\usepackage{ragged2e}
\usepackage{listings}
\usepackage[a4paper, total={6in, 8in}]{geometry}

\newcommand{\Z}{\mathbb{Z}}
\newcommand{\N}{\mathbb{N}}

\begin{document}
	\centering
	\LARGE\textbf{Tutorial \#1: Introduction to Algebra}
	
	\justifying
	\section{Exercise 1:}
		\Large\textbf{Proof:}
		\begin{equation}
			\sum_{i=1}^{n} i^2 = \frac{(n)(n+1)(2n+1)}{6}\\
		\end{equation}
		\Large\textbf{Base case:}
		When $n = 1$, the left hand side (LHS) of (1) is $1^2$ = 1, and the right hand side (RHS) of (1) is $\frac{1\times(1 + 1)\times(2\times1 + 1)}{6} = 1$. Thus, (1) is true for $n=1$\\
		\Large\textbf{Induction step:}\\
		Let $k \in \N$ be given and suppose (1) is true for $n = k.$ Then:\\
		\begin{align*}
			\sum_{i=1}^{k+1} i^2 &= \frac{(k+1)(k+2)(2k+3)}{6}\\
			\sum_{i=1}^{k} i^2 + (k+1)^2 &= \frac{(k+1)(k+2)(2k+3)}{6}\\
			\frac{(k)(k+1)(2k+1)}{6} + (k+1)^2 &= \frac{(k+1)(k+2)(2k+3)}{6}\\
			\frac{(k+1)(2k^2+k+6(k+1))}{6} &= \frac{(k+1)(k+2)(2k+3)}{6}\\
			\frac{(k+1)(2k^2+7k+6)}{6} &= \frac{(k+1)(k+2)(2k+3)}{6}\\
		\end{align*}
		Thus, (1) holds to $n = k + 1$ and the proof of induction step is complete.\\
		\Large\textbf{Conclusion:} By the principle of induction, (1) is true for all $n \in \N$

	\section{Exercise 2:}
		\Large\textbf{Proof:}
		\begin{equation}
			n! > 2^n \; for \; n \ge 4
		\end{equation}
		\Large\textbf{Base case:}
		When $n=4$, $LHS = 4! = 24, RHS = 2^4 = 16.$ Thus, (2) holds for $n=4$.\\
		\Large\textbf{Induction step:}
		Let $k \in \N$ be given and suppose (2) is true for $n = k.$ Then:\\
		\begin{align*}
			(k+1)! > 2^{k+1}\\
			(k+1)! = k!\times(k+1) > 2k! > 2^{k+1}
		\end{align*}
		Thus, (2) holds to $n=k+1$ and the proof of induction step is complete.\\
		\Large\textbf{Conclusion:} By the principle of induction, (2) is true for all $n \ge 4$
	
	\section{Exercise 3:}
		\Large\textbf{Proof:} $10^{n+1}  + 10^n + 1$ is divisible by 3 $\forall n \in \N$.\\
		As $10^{n+1}  + 10^n + 1$ is represented as $11000.....001$. Thus, the total of digit values is 3, which is divisible for 3. Thus, $10^{n+1}  + 10^n + 1$ is divisible by 3 $\forall n \in \N$.
		
	\section{Exercise 4:}
		\Large\textbf{Proof:}
		\begin{equation}
			\sum_{i=0}^{n} 2^n = 2^{n+1} -1
		\end{equation}
		\Large\textbf{Base case:} When $n=1$, LHS = $2^0 + 2^1 = 3$, RHS = $2^{1+1} -1 = 3$. Thus, (3) holds for $n=1$.\\
		\Large\textbf{Induction step:}
		Let $k \in N$ be given and suppose (2) is true for $n = k.$ Then:\\
		\begin{align*}
			\sum_{i=0}^{k+1} 2^i &= 2^{k+2} - 1\\
			\sum_{i=0}^{k} 2^i + 2^{k+1} &= 2^{k+2} - 1\\
			2^{k+1} - 1 + 2^{k+1} &= 2^{k+2} -1\\
			2^{k+1} + 2^{k+1} &= 2^{k+2}
		\end{align*}
		Thus, (3) holds for $n=k+1$ and the proof of induction step is complete.
		\Large\textbf{Conclusion:} By the principle of induction, (3) is true $\forall n \in \N$
	
	\section{Exercise 5:}
		\begin{enumerate}
			\item[a)]
				\begin{align*}
					3x \equiv 2 \;(mod \; 7)\\
					3x\times5 \equiv 2\times5 \; (mod \; 7)\\
					x \equiv 3 \; (mod \; 7)
				\end{align*}
			\item[b)]
				\begin{align*}
					5x + 1 \equiv 13 \; (mod \; 23)\\
					5x \equiv 12 \; (mod \; 23)\\
					5x\times14 \equiv 12\times14 \; (mod \; 23)\\
					x \equiv 7 \; (mod \; 23)
				\end{align*}
			\item[c]
				\begin{align*}
					2x \equiv 1 \; (mod \; 6)\\
				\end{align*}
				Since there are not exist such $k$ that $2k=1$, (c) has not solution.
		\end{enumerate}
	
	\section{Exercise 6:}
		\centering
		\begin{math}
			\begin{cases}
				3x + 7y \equiv 4 \mbox{ (mod 11)}\\
				8x + 6y \equiv 1 \mbox{ (mod 11)}
			\end{cases}
		\end{math}
		
		\justify
		Cayley table for $\Z_{11}$\\
		
		\centering
		\begin{tabular} {| c |c c c c c c c c c c |}
			\hline
			$\Z_{11}$ & 1 & 2 & 3 & 4 & 5 & 6 & 7 & 8 & 9 & 10 \\
			\hline
			1 & 1 & 2 & 3 & 4 & 5 & 6 & 7 & 8 & 9 & 10\\
			2 & 2 & 4 & 6 & 8 & 10 & 1 & 3 & 5 & 7 & 9\\
			3 & 3 & 6 & 9 & 1 & 4 & 7 & 10 & 2 & 5 & 8\\
			4 & 4 & 8 & 1 & 5 & 9 & 2 & 6 & 10 & 3 & 7\\
			5 & 5 & 10 & 4 & 9 & 3 & 8 & 2 & 7 & 1 & 6\\
			6 & 6 & 1 & 7 & 2 & 8 & 3 & 9 & 4 & 10 & 5\\
			7 & 7 & 3 & 10 & 6 & 2 & 9 & 5 & 1 & 8 & 4\\
			8 & 8 & 5 & 2 & 10 & 7 & 4 & 1 & 9 & 6 & 3\\
			9 & 9 & 7 & 5 & 3 & 1 & 10 & 8 & 6 & 4 & 2\\
			10 & 10 & 9 & 8 & 7 & 6 & 5 & 4 & 3 & 2 & 1\\
			\hline
		\end{tabular}
		
		\justify
		\begin{math}
			\begin{cases}
				3x + 7y \equiv 4 \mbox{ (mod 11)}\\
				8x + 6y \equiv 1 \mbox{ (mod 11)}
			\end{cases}
			\iff
			\begin{cases}
				8x + 4y \equiv 4 \mbox{ (mod 11)}\\
				8x + 6y \equiv 1 \mbox{ (mod 11)}
			\end{cases}
			\\\\\\
			\iff
			\begin{cases}
				8x + 4y \equiv 4 \mbox{ (mod 11)}\\
				2y \equiv 8 \mbox{ (mod 11)}
			\end{cases}
			\iff
			\begin{cases}
				8x + 4\times4 \equiv 4 \mbox{ (mod 11)}\\
				y \equiv 4 \mbox{ (mod 11)})
			\end{cases}
			\\\\\\
			\iff
			\begin{cases}
				8x \equiv 10 \mbox{ (mod 11)}\\
				y \equiv 4 \mbox{ (mod 11)}
			\end{cases}
			\iff
			\begin{cases}
				x \equiv 4 \mbox{ (mod 11)}\\
				y \equiv 4 \mbox{ (mod 11)}
			\end{cases}
		\end{math}
	
	\section{Exercise 7:}
		\begin{math}
			\begin{cases}
				3x + 5y - 7z \equiv 8 \mbox{ (mod 83)}\\
				8x - 9y + 13z \equiv 13 \mbox{ (mod 83)}\\
				7x + 4y + 5z \equiv 15 \mbox{ (mod 83)}
			\end{cases}
			\iff
			\begin{cases}
				x + 37y - 2z \equiv 26 \mbox{ (mod 83) ($\times$24)}\\
				8x - 9y + 13z \equiv 13 \mbox{ (mod 83)}\\
				7x + 4y + 5z \equiv 15 \mbox{ (mod 83)}
			\end{cases}
			\\\\\\
			\iff
			\begin{cases}
				x + 37y - 2z \equiv 26 \mbox{ (mod 83)}\\
				-56y + 29z \equiv -29 \mbox{ (mod 83)}\\
				-6y + 19z \equiv -1 \mbox{ (mod 83)}
			\end{cases}
			\iff
			\begin{cases}
				x + 37y - 2z \equiv 26 \mbox{ (mod 83)}\\
				27y + 29z \equiv -29 \mbox{ (mod 83)}\\
				-6y + 19z \equiv -1 \mbox{ (mod 83)}				
			\end{cases}
			\\\\\\
			\iff
			\begin{cases}
				x + 37y - 2z \equiv 26 \mbox{ (mod 83)}\\
				y + 81z \equiv 2 \mbox{ (mod 83) ($\times 40)$}\\
				-6y + 19z \equiv -1 \mbox{ (mod 83)}				
			\end{cases}
			\iff
			\begin{cases}
				x + 37y - 2z \equiv 26 \mbox{ (mod 83)}\\
				y + 81z \equiv 2 \mbox{ (mod 83)}\\
				7z \equiv 15 \mbox{ (mod 83)}
			\end{cases}
			\\\\\\
			\iff
			\begin{cases}
				x + 37y - 2z \equiv 26 \mbox{ (mod 83)}\\
				y + 81z \equiv 2 \mbox{ (mod 83)}\\
				z \equiv 14 \mbox{ (mod 83)}
			\end{cases}
			\iff
			\begin{cases}
				x \equiv 23 \mbox{ (mod 83)}\\
				y \equiv 30 \mbox{ (mod 83)}\\
				z \equiv 14 \mbox{ (mod 83)}
			\end{cases}
		\end{math}
			Any mistakes might be considered later ;)
	
	\section{Exercise 8:}
		Cayley table for $\Z_7$
	
		\centering
		\begin{tabular} {| c |c c c c c c |}
			\hline
			$\Z_7$ & 1 & 2 & 3 & 4 & 5 & 6\\
			\hline
			1 		  & 1 & 2 & 3 & 4 & 5 & 6\\
			2 		  & 2 & 4 & 6 & 1 & 3 & 5\\
			3 		  & 3 & 6 & 2 & 5 & 1 & 4\\
			4 		  & 4 & 1 & 5 & 2 & 6 & 3\\
			5 		  & 5 & 3 & 1 & 6 & 4 & 2\\
			6 		  & 6 & 5 & 4 & 3 & 2 & 1\\
			\hline
		\end{tabular}
		
		\justify
		\begin{math}
			A = \begin{bmatrix}
					5 & 2\\
					6 & 3
				\end{bmatrix}
			\Rightarrow
			[A|I] =	\begin{bmatrix}
				  		5 & 2 & 1 & 0\\
						6 & 3 & 0 & 1
			  		\end{bmatrix}
			  	  = \begin{bmatrix}
			  	  		1 & 6 & 3 & 0\\
			  	  		6 & 3 & 0 & 1
			  	  \end{bmatrix}
			  	  = \begin{bmatrix}
			  	  		1 & 6 & 3 & 0\\
			  	  		0 & 2 & 3 & 1
			  	  \end{bmatrix}
			  	  \\\\\\
			  	  = \begin{bmatrix}
					  	1 & 6 & 3 & 0\\
					  	0 & 1 & 5 & 4
			  	  \end{bmatrix}
			  	  = \begin{bmatrix}
				  	    1 & 0 & 1 & 4\\
						0 & 1 & 5 & 4\\					
			  	  \end{bmatrix}
			  	  = [I|A^{-1}]
		\end{math}\\\\\\
		Thus, the inverse of $A$ is $A^{-1} = \begin{bmatrix}
												1 & 0 & 1 & 4\\
												0 & 1 & 5 & 4	
											  \end{bmatrix}$
\end{document}