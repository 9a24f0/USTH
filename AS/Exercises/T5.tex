\documentclass{article}
\usepackage[utf8]{inputenc}
\usepackage{mathtools}
\usepackage{amssymb}
\usepackage{amsmath}
\usepackage{graphicx}
\usepackage{ragged2e}
\usepackage{listings}
\usepackage[a4paper, total={6in, 8in}]{geometry}

\setcounter{secnumdepth}{0}

\newcommand{\Z}{\mathbb{Z}}
\newcommand{\N}{\mathbb{N}}
\newcommand{\Q}{\mathbb{Q}}
\newcommand{\R}{\mathbb{R}}


\begin{document}
	\centering
	\LARGE\textbf{Tutorial \#5: Rings and Fields}\\
	
	\justify\Large
	
	\section{Exercise 0:}
		\begin{enumerate}
			\item [a)]
				Since $a0 = a(0+0) = a0 + a0$. Thus, $a0 = 0$.\\
				Since $0a = (0 + 0)a = 0a + 0a$. Thus, $0a = 0$.\\
				Therefore, $a0 = 0a = 0$.
			\item [b)]
				We have:
					\begin{align*}
						ab + (-a)b &= (a + (-a))b = 0b = 0.\\
						ab + a(-b) &= a(b + (-b)) = a0 = 0.\\
						ab - ab &= ab + (-ab) = 0\\						
					\end{align*}
				The inverse of $ab$ is unique as R is a abelian group under addition. Thus, $(-a)b = a(-b) = -ab$.
			\item [c)]
				We have:
					\begin{align*}
						(-a)(-b) + (-a)b &= (-a)(-b + b) = 0.\\
						ab + (-a)b &= ab - ab = 0.
					\end{align*}
				The inverse of $(-a)b$ is unique as R is an abelian group under addition operation. Hence, $(-a)(-b) = ab$.
		\end{enumerate}
	
	\section{Exercise 1:}
		\begin{itemize}
			\item 7$\Z$ \\
				\begin{enumerate}
					\item [i) ]
						Let $a, b, c \in 7\Z$, $\exists i, j$ and $k \in \Z$ such that $a = 7i, b = 7j, c = 7k$.
						\begin{align*}
							(a + b) + c &= (7i + 7j) + 7k = 7(i+j) + 7k = 7(i+j+k).\\
							a + (b + c) &= 7i + (7j + 7k) = 7i + 7(j+k) = 7(i+j+k).
						\end{align*}
						Hence, $(a + b) + c = a + (b + c)$, the ring is associative under addition operation.
					\item [ii)]
						There exist $e=0$ such that $a + e = 7i + 0 = 7i = a$. Thus, ring R has identity element under addition operation.
					\item [iii)]
						There exist $a^{-1} = -7i$ such that $a + a^{-1} = 7i - 7i = 0 = e$. ring R has inverse element under addition operation.
					\item [iv)]
						Since $a + b = 7(i+j) = 7(j+i) = b + a$, ring R is commutative under addition operation.
					\item [v)]
						Since $ab = 7i\times7j = 7j\times7i = ba$, ring R is commutative under multiplication operation.
					\item [vi)]
						Since $(a+b)c = (7i+7j)7k = 49ik + 49jk = ac + bc$, ring R is multiplicative distributive associated with addition operation.
				\end{enumerate}
				\textbf{Conclusion:} Hence, $7\Z$ is a ring.
			\item $\Q(\sqrt{2}) = \{a+b\sqrt{2} \; \vert \; a, b \in \Q\}.$\\
				\begin{enumerate}
					\item [i)]
						Let $a, b, c \in \Q(\sqrt{2}), \exists i,j,k,l,m,n$ such that:
						\begin{math}
							\begin{cases}
								a = i + \sqrt{2}l\\
								b = j + \sqrt{2}m\\
								c = k + \sqrt{2}n
							\end{cases}
						\end{math}\\
						$(a+b)+c = (i+j+\sqrt{2}(l+m))+k+\sqrt{2}n=i+j+k+ \sqrt{2}(l+m+n)\\
						a + (b + c) = i + \sqrt{2}l + (j + k+ \sqrt{2}(m + n)) = i + j + k + \sqrt{2}(l+m+n)$
						Hence, ring R is associative under addition operation.
					\item [ii)]
						There exists $e = 0$ such that $a+e = (i+\sqrt{2}l) + 0 = i+\sqrt{2}l = a$. Thus, ring R has identity element under addition operation.
					\item [iii)]
						There exists such $a^{-1} = -i-\sqrt{2}l$ satisfies $a + a^{-1} = e$. Thus ring R has inverse element under addition operation.
					\item [iv)]
						Since $a+b = i+j+\sqrt{2}(l+m) = j+i + \sqrt{2}(m+l) = b + a$. Thus, ring R is commutative under addition operation.
					\item [v)]
						Since $ab = (i+\sqrt{2}l)(j+\sqrt{2}m) = ij + \sqrt{2}(lj+im) +2lm = ji + \sqrt{2}(mi + jl) + 2ml = (j+\sqrt{2}m)(i+\sqrt{2}l) = ba$, ring R is commutative under multiplication operation.
					\item [vi)]
						Since $(a+b)c=(i+\sqrt{2}l+j\sqrt{2}m)(k+\sqrt{2}n) = ... = ab + ac$, ring R is multiplicative distributive associated with addition operation.
				\end{enumerate}
		\end{itemize}
	
	\section{Exercise 2:}
		\textbf{Prove that:} R is a commutative ring. Given that R is a ring and $a^2 = a$ for every $ a \in \R$.\\
		\textbf{Proof:}\\
		With $a = a^2 (\forall a \in R)$, we can prove that:
		\begin{align*}
			a + a &= (a + a)^2\\
			a + a &= a^2 + a^2 + a^2 + a^2\\
			a + a &= a + a + a + a\\
			a + a &= 0\\
			a &= -a
		\end{align*}
		Let $a, b \in$ R, since R is a ring, R is also an abelian group under addition operation. Therefore, $(a+b) \in$ R,
		\begin{align*}
			(a+b)^2 &= a+b\\
			a^2 + ab + ba + b^2 &= a^2 + b^2\\
			ab &= -ba		
		\end{align*}
		Since $ b = -b, \forall b \in$ R, we can conclude that $ab = ba$. Hence, ring R is commutative.
	
	\section{Exercise 3:}
		\textbf{Prove that:} $\phi(1)$ is identity element of R' if R is a ring with identity element 1 and $\phi$ is a homomorphism of R onto R'.\\
		\textbf{Proof:} Since $\phi$ is homomorphism, we have: $\phi(ab) = \phi(a)\phi(b)$\\
		Let $x \in R$ and $1_R$ be identity element of R, we have $1_R * x = x \; \forall x \in R$. 
		\begin{equation*}
			\phi(x) = \phi(1_R*x) = \phi(1_R)\phi(x)
		\end{equation*}
		Thus, $\phi(1_R)$ is the identity element of R'.
	\section{Exercise 4:}
		\textbf{Prove that:} Z(R) is a subring of R, Z(R) = $\{x \in \R \vert xy = yx, \forall y \in \R\}$
		\begin{enumerate}
			\item [i)]
				It's trivial that Z(R) $\neq \emptyset$
			\item [ii)]
				Let $a, b \in$ Z(R), thus $at = ta, bt = tb, \; \forall t \in \R$.
				\begin{align*}
					atb &= atb\\
					(at)b &= a(tb)\\
					(ta)b &= a(bt)\\
					t(ab) &= (ab)t.
				\end{align*}
				Thus, $ab \in Z(R)$.
			\item [iii)]
				Since $a, b \in$ Z(R),
				\begin{align*}
					at - bt &= at - bt\\
					at + (-b)t &= ta + t(-b)\\
					(a+(-b))t &= t(a + (-b))\\
					(a-b)t &= t(a-b)
				\end{align*}
				Thus, $(a-b) \in Z(R)$.\\
		\end{enumerate}
		\textbf{Conclusion:} By the definition of subring, Z(R) is a subring of $\R$.
		
	\section{Exercise 5:}
		\textbf{General formula:} $1 + (-1)^{n-1}x^n = (1+x)\sum_{i=0}^{n-1}(-x)^i$\\
		It's trivial that with $n \equiv 1 \pmod 2$,
		\begin{equation*}
			1 = x^n + 1 = x^n(-1)^{n-1} + 1 = (x+1)\sum_{i=0}^{n-1} (-x)^i
		\end{equation*}
		Since $x \in \R \mbox{ and } i \in \N$, there exists $\sum_{i=0}^{n-1} (-x)^i \in \R$. Thus, $(x+1)$ is an unit.
		
	\section{Exercise 6:}
		\textbf{Prove that:} If a ring is isomorphic to a field, then that ring is a field.\\
		\textbf{Proof:} Let $f: R \rightarrow F$ be an isomorphism from ring R to field F.
		\begin{enumerate}
			\item [i)]
				Let $a, b \in$ R, since R is a ring, $ab \in$ R. Since F is a field, F is multiplicative commutative ($f(a)f(b)=f(b)f(a)$). Thus, by the definition of homomorphism, we have:
				\begin{equation*}
					f(ab) = f(a)f(b) = f(b)f(a) = f(ba)
				\end{equation*}
				Since $f$ is injective, we can conclude that $ab = ba$ and ring R is commutative under multiplication.\\
			\item [ii)]
				Let $1_F$ be multiplicative identity of F, $\forall a \in$ R,and $f$ is isomorphism:
				\begin{equation*}
					f^{-1}(1_F)*a = f^{-1}(1_F)*f^{-1}(f(a)) = f^{-1}(1_F*f(a)) = f^{-1}(f(a)) = a
				\end{equation*}
				Thus, $1_R = f^{-1}(1_F)$ is multiplicative identity of R.
			\item [iii)]
				Since F is a field, $\forall a \neq 0 \in$ R, we have:
				\begin{align*}
					f^{-1}(a) * f(a) &= e\\
					\frac{1_F}{f(a)}f(a) &= 1_F\\
					f^{-1}\left(\frac{1_F}{f(a)}f(a)\right) &= 1_R\\
					f^{-1}\left(\frac{1_F}{f(a)}\right)f^{-1}(f(a)) &= 1_R\\
					f^{-1}\left(\frac{1_F}{f(a)}\right)a &= 1_R\\
				\end{align*}
				Thus, $\frac{1_R}{a} = f^{-1}\left(\frac{1_F}{f(a)}\right)$ is multiplicative inverse of R.
		\end{enumerate}
		\textbf{Conclusion:} With i), ii) and iii), we conclude that R is also a field.
	
	\section{Exercise 7:}
		\textbf{Prove that:} $(g \circ f) : A \rightarrow C$ is isomorphism if $f: A \rightarrow B$ and $g: B \rightarrow C$ are isomorphisms.\\
		$f$ and $g$ are isomorphisms $\rightarrowtail
		\begin{cases}
			f(a)f(b) &= f(ab)\\
			g(a)g(b) &= g(ab)
		\end{cases}$ 
		Thus, we have:
		\begin{equation*}
		(f \circ g)(a) * (f \circ g)(b) = f(g(a))*f(g(b)) = f(g(a)*g(b)) = f(g(ab)) = (f \circ g)(ab)
		\end{equation*}
		In addition, since both $f$ and $g$ is bijective, $f \circ g$ is also bijective.
		\textbf{Conclusion:} $f \circ g$ is isomorphism.
\end{document}