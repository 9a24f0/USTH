\documentclass{article}
\usepackage[utf8]{inputenc}
\usepackage{mathtools}
\usepackage{amssymb}
\usepackage{graphicx}
\usepackage{ragged2e}
\usepackage{listings}
\usepackage[a4paper, total={6in, 8in}]{geometry}

\newcommand{\Z}{\mathbb{Z}}
\newcommand{\N}{\mathbb{N}}
\newcommand{\Q}{\mathbb{Q}}
\newcommand{\R}{\mathbb{R}}

\begin{document}
	\centering
	\LARGE\textbf{Tutorial \#3.2: Cyclic Group}
	
	\justify\Large
	\section{Exercise 1:}
	All cyclic subgroups of $G = (\Z_7, *)$
		\begin{align*}
			<1> &= \{1\}\\
			<2> &= \{1,2,4\}\\
			<3> &= \{1,2,3,4,5,6\}\\
			<4> &= \{1,2,4\}\\
			<5> &= \{1,2,3,4,5,6\}\\
			<6> &= \{1,6\}
		\end{align*}
	
	\section{Exercise 2:}
		Let a be the generator of cyclic group G. Thus, $G = \;<a>$\\
		\begin{enumerate}
			\item [a)] G be a cyclic group of order 6 ($C_6 = \{1, a, a^2, a^3, a^4, a^5\}$). There only $a$ and $a^5$ generates $C_6$.
			\item [b)] G be a cyclic group of order 5 ($C_5 = \{1, a, a^2, a^3, a^4\}$). All elements excepts 1 can generate $C_5$.
			\item [c)] G be a cyclic group of order 8 ($C_8 = \{1, a, a^2, a^3, a^4, a^5, a^6, a^7\}$). There are $a, a^3, a^5, a^7$ generates $C_8$
			\item  [d)] G be a cyclic group of order 10 ($C_{10} = \{1, a, a^2, a^3, a^4, a^5, a^6, a^7, a^8, a^9\}$). There are $a, a^3, a^7, a^9$ generates $C_{10}$  
		\end{enumerate}
		
	\section{Exercise 3:}
		\textbf{In general,} to determines if $G$ is a cyclic group, we must find a generator $g$ such $<g> \; = G$. $g$ is an element in $G$ and is a co-prime.
		For example, to determines whether if $Z_6^*$ is a cyclic group, we only needs to test $g \in \{1, 5\}$ 
		\begin{enumerate}
			\item [a)] $G = \Z_7^* = \; <3>$. Thus, $G$ is cyclic
			\item [b)] $G = \Z_{12}^*$. Since there exists no such $g \in \{1,5,7,11\}$ that generates $\Z_{12}^*$, $G$ is not cyclic.
		\end{enumerate}
	
	\section{Exercise 4:}
		\begin{enumerate}
			\item [a)] $U(18) = \{1, 5, 7, 11, 13, 17\}$. The subgroup generated by 5 in $U(18)$ is $\{1,5,7,11,13,17\}$.
			\item [b)] $U(20) = \{1,3,5,7,9,11,13,17,19\}$. The subgroup generated by 3 in $U(20)$ is $\{1,3,7,9\}$
		\end{enumerate}
	
	\section{Exercise 5:}
		\begin{enumerate}
			\item [a)]
				As $<3> = \{1,3,7,9\}$, $<3>$ is a cyclic subgroup of order 4 in $G = Z_{20}^*$.
			\item [b)]
				Let $G_{non-cyclic}$ be $\{1,2,3,4\}$ is a subgroup of $Z_{20}^*$. While possible generator of $G_{non-cyclic}$ is $g \in \{1,2,3,4\}$, there exist no such $g$ generates $G_{non-cyclic}$. Thus, $G_{non-cyclic}$ is a non-cyclic subgroup of $G$
		\end{enumerate}

	\section{Exercise 6:}
		\textbf{Prove that:} $b$ is generator of $G$. Given that $G= \; <a>$ and $b \in G$ such that $a = b^k$.\\
		\textbf{Proof:} As $G = \; <a>$, it's trivial that $G = \{1, a, a^2, ... a^n\}$. While we having $a = b^k$, we can derive $G$ as $G = \{1, b^k, b^{2k},..., b^{nk}\}$. Hence, $b$ is a generator of $G$.
\end{document}