\documentclass{article}
\usepackage[utf8]{inputenc}
\usepackage{mathtools}
\usepackage{amssymb}
\usepackage{graphicx}
\usepackage{ragged2e}
\usepackage{listings}
\usepackage[a4paper, total={6in, 8in}]{geometry}

\setcounter{secnumdepth}{0}

\newcommand{\Z}{\mathbb{Z}}
\newcommand{\N}{\mathbb{N}}
\newcommand{\Q}{\mathbb{Q}}
\newcommand{\R}{\mathbb{R}}

\begin{document}
	\centering
	\LARGE\textbf{Midterm 2019}\\
	\Large Special thanks to BI8 has given us this :* 
	
	\justify
	
	\section{Problem 1.1:}
		\textbf{Prove that:} For $n \in \N$,
			\begin{equation}
				\sum_{j=1}^{n} j^3 = \frac{1}{4}n^2(n+1)^2
			\end{equation}
		\textbf{Base case:} When $n = 1, LHS = 1^3 = 1, RHS = \frac{1}{4}1^2(1+1)^2 = 1$. Thus, (1) holds for $n=1$.\\
		\textbf{Induction step:} Let $k \in \N$ be given and suppose that (1) is true for $n=k$. Then:\\
		\begin{align*}
			\sum_{j=1}^{k+1} j^3 &= \frac{1}{4}(k+1)^2(k+2)^2\\
			\sum_{j=1}^{k} j^3 + (k+1)^3 &= \frac{1}{4}(k+1)^2(k+2)^2\\
			\frac{1}{4}k^2(k+1)^2 + (k+1)^3 &= \frac{1}{4}(k+1)^2(k+2)^2\\
			(k+1)^2(k^2 + 4k + 4) &= (k+1)^2(k+2)^2\\
		\end{align*}
		Thus, (1) holds for $n=k+1$, the proof of induction step is complete.
		\textbf{Conclusion:} By principle of induction, (1) is true for all $n \in \N$.
	
	\section{Problem 2.1:}
		\begin{math}
			\begin{cases*}
				3x + 5y \equiv 14 \mbox{ (mod 17)}\\
				7x + 3y \equiv 6 \mbox{ (mod 17)}
			\end{cases*}
			\iff
			\begin{cases*}
				x + 13y \equiv 16 \mbox{ (mod 17)}\\
				7x + 3y \equiv 6 \mbox{ (mod 17)}
			\end{cases*}
			\\\\\\
			\iff
			\begin{cases*}
				x + 13y \equiv 16 \mbox{ (mod 17)}\\
				3y \equiv 4 \mbox{ (mod 17)}
			\end{cases*}
			\iff
			\begin{cases*}
				x \equiv 10 \mbox{ (mod 17)}\\
				y \equiv 7 \mbox{ (mod 17)}
			\end{cases*}
		\end{math}
		
	\section{Problem 3.1:}
		\begin{equation*}
			f(x) = \sqrt{x^3-7}
		\end{equation*}
		$f(x) \mbox{ has domain }D = [\sqrt[3]{7}, +\infty), \mbox{ range } R = [0, +\infty)$\\
		Let $y = f(x)$, we can re-write the function as:
		\begin{equation*}
			y = \sqrt{x^3-7} \iff x = \sqrt[3]{y^2 + 7}
		\end{equation*}
		Thus, the inverse function of $f(x)$ is $f^{-1} = \sqrt[3]{x^2 + 7}$\\
		\begin{equation}
			f \circ f^{-1} = f(f^{-1}(x)) = \sqrt{(\sqrt[3]{x^2+7})^3-7} = |x| = x
		\end{equation}
		\begin{equation}
				f^{-1} \circ f = f^{-1}(f(x)) = \sqrt[3]{(\sqrt{x^3-7})^2 + 7} = x
		\end{equation}
		\textbf{Conclusion:} Since (2) = (3) with $x \in D$, $f \circ f^{-1} = f^{-1} \circ f \; \forall x \in D$.
		
	\section{Problem 1.2:}
		\textbf{Prove that:} For $n \in \N$,
		\begin{equation*}
						\sum_{i=1}^{n} \frac{1}{i(i+1)} = \frac{n}{n+1}
		\end{equation*}
		Since i'm sleepy and this is such trivial that I proved it in $5^{th}$ grade, reader can prove it himself/herself.
	
	\section{Problem 2.2:}
		\begin{math}
			\begin{cases*}
				5x - 6y \equiv 9 \mbox{ (mod 22)}\\
				8x + y \equiv 12 \mbox{ (mod 22)}
			\end{cases*}
			\iff
			\begin{cases*}
				x - 10y \equiv 15 \mbox{ (mod 22)}\\
				8x + y \equiv 12 \mbox{ (mod 22)}
			\end{cases*}
			\\\\\\
			\iff
			\begin{cases*}
				x - 10y \equiv 15 \mbox{ (mod 22)}\\
				15y \equiv 2 \mbox{ (mod 22)}
			\end{cases*}
			\iff
			\begin{cases*}
				x \equiv 9 \mbox{ (mod 22)}\\
				y \equiv 6 \mbox{ (mod 22)}
			\end{cases*}
		\end{math}
		
	\section{Problem 3.2:}
		\begin{equation*}
			f(x)= x^2 - 3x + 2
		\end{equation*}
		$f(x)$ has domain $D = \R$, range $R = (-\frac{1}{4}, +\infty)$.\\
		Let $f(x) = y$, we can re-write the function as:
		\begin{align*}
			y = x^2 -3x+2 &\iff 0 = x^2 - 3x + (2-y)
			&\iff x = \frac{3 \; \pm \sqrt{3^2 - (2 -y)}}{2}
		\end{align*}
		Thus, the inverse function of $f(x)$ is $f^{-1} = \frac{3 \; \pm \sqrt{y+7}}{2}$.\\
		We can simple prove that $f \circ f^{-1} = f^{-1} \circ f = x$ by using Linear Algebra which is already learned in high school ;).
\end{document}