\documentclass{article}
\usepackage[utf8]{inputenc}
\usepackage{mathtools}
\usepackage{amssymb}
\usepackage{amsmath}
\usepackage{graphicx}
\usepackage{ragged2e}
\usepackage{listings}
\usepackage[a4paper, total={6in, 8in}]{geometry}

\setcounter{secnumdepth}{0}

\newcommand{\Z}{\mathbb{Z}}
\newcommand{\N}{\mathbb{N}}
\newcommand{\Q}{\mathbb{Q}}
\newcommand{\R}{\mathbb{R}}

\begin{document}
	\centering
	\LARGE\textbf{Tutorial \#3.4: Cosets}\\
	
	\justify\Large
	
	\section{Exercise 1:}
		\begin{itemize}
			\item $<8>$ in $(\Z_{24}, +)$\\
			Since $<8> = \{0, 8, 16\}$, the left cosets of $<8>$ in $\Z_{24}$ is
			\begin{align*}
				0 + <8> &= \{0, 8, 16\} = 8 + <8> = 16 + <8>\\
				1 + <8> &= \{1, 9, 17\} = 9 + <8> = 17 + <8>\\
				2 + <8> &= \{2, 10, 18\} = 10 + <8> = 18 + <8>\\
				3 + <8> &= \{3, 11, 19\} = 11 + <8> = 19 + <8>\\
				4 + <8> &= \{4, 12, 20\} = 12 + <8> = 20 + <8>\\
				5 + <8> &= \{5, 13, 21\} = 13 + <8> = 21 + <8>\\
				6 + <8> &= \{6, 14, 22\} = 14 + <8> = 22 + <8>\\
				7 + <8> &= \{7, 15, 23\} = 15 + <8> = 23 + <8>\\
			\end{align*}
			It's trivial that $(\Z_{24}, +)$ is an abelian groanan. Thus $(\Z_{24}, +)$ is commutative and the right cosets is the same as above.
		\end{itemize}
		\begin{itemize}
			\item $<3>$ in $(U(8), .)$\\
			Since $<3> =\{1,3\}, (U(8), .) = \{1,3,5,7\}$, the cosets of $<3>$ in $(U(8), .)$ is
			\begin{align*}
				1 * <3> &= \{1, 3\} = 3 * <3>\\
				5 * <3> &= \{5, 7\} = 7 * <3>
			\end{align*}
			It's trivial that $(U(8), .)$ is an abelian group. Thus $(U(8), .)$ is commutative and the right cosets is the same as above.
		\end{itemize}
	
	\newpage
	\section{Exercise 2:}
		Let $k$ be the order of group G. Assume that the order $k$ does not divides $n$. That we have: $n = kd + r$ with $d, r\in \N, 0 < r < k$.
		Since $g^n = e \iff g^{kd+r} = e \iff g^{kd}g^r = e \iff eg^r = e \iff g^r =e$, this contradicts with the property of order of the group.\\
		\textbf{Conclusion:} The order of group $k$ must divides $n$.
	
	\section{Exercise 3:}
		$H = \{3k, k \in \N \}$, the left cosets of $(H, +)$ in $\Z$ are:
		\begin{align*}
			0 + H &= \{3k, k \in \N \} &&= 3d + H && \forall d \in \Z\\
			1 + H &= \{3k+1, k \in \N \} &&= 3d+1 + H && \forall d \in \Z\\
			2 + H &= \{3k+2, k \in \N \} &&=  3d+2 + H && \forall d \in \Z
		\end{align*}
		\begin{enumerate}
			\item 11 + H and 17 + H\\
			Since 11 and 17 $\in \{3d+1, d \in \Z \}$, these two cosets are equivalent.
			\item -1 + H and 5 + H\\
			Since -1 and 5 $\in \{3d+2, d \in \Z \}$, these two cosets are equivalent.
			\item 7 + H and 23 + H\\
			Since 7 + H $\in \{3d+1, d \in \Z \}$, 23 + H $\in \{3d+2, d \in \Z \}$, these two cosets are not equivalent.
		\end{enumerate}
	
	\section{Exercise 4:}
		$G$ is a group order by 15 and is generated by $a$. Thus, $ G = \{a^1, a^2,...a^{14}, e\}$.\\
		Left cosets of $<a^5>$ in $<a>$ are:
		\begin{align*}
			e * <a^5> &= \{e, a^5, a^{10}\}\\
			a * <a^5> &= \{a, a^6, a^{11}\}\\
			a^2 * <a^5> &= \{a^2, a^7, a^{12}\}\\
			a^3 * <a^5> &= \{a^3, a^8, a^{13}\}\\
			a^4 * <a^5> &= \{a^4, a^9, a^{14}\}
		\end{align*}
	
	\section{Exercise 5:}
		Since $G$ is a group of order 60, let $a \in G$, we have $a^{60} = e$ following the property of order of a group.\\
		Let $k$ be the order of the subgroup of G. Thus, $a^k = e, \; k \in \N$.\\
		Since $e^n = e \; \forall n \in \N$ and $a^k = e, a^{60} = e$, there exist such $n \in \N$ satisfies that $kn = 60$. Hence, $k$ must divides $n$.\\
		\textbf{Conclusion: } $k \in \{1,2,3,4,5,6,10,12,15,20,30,60\}$
\end{document}